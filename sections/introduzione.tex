\section{Introduzione}

Il presente documento ha lo scopo di descrivere l'attività di stage
svolta dallo studente presso l'azienda Innove di Thiene (Vicenza).

\subsection{Lo stage}
Scopo dell'attività di stage è stato la realizzazione della versione
mobile (da qui in poi chiamata \fiscoloMobile) di un'applicazione web 
già esistente (da qui in poi chiamata \fiscoloWeb) utilizzando il
framework \textit{Phonegap} e diverse altre tecnologie innovative.

\subsection[\fiscoloWeb]{\fiscoloWeb\footnote{\texttt{ http://www.fiscolo.it/}}}
L'applicazione web esistente è indirizzata principalmente a professionisti
e piccole aziende ed è divisa in due moduli integrati che offrono diverse 
funzionalità, è gratuita ed offre la possibilità di passare ad una versione
a pagamento per ottenere l'accesso a funzionalità aggiuntive. \\
L'applicazione è sviluppata in Java e Scala tramite il framework
Play\footnote{\texttt{ https://www.playframework.com/}}.

\subsubsection{\fiscolo}
Questo modulo si occupa di contabilità, tra le funzionalità offerte figurano:

\begin{itemize}
\item Possibilità di creare fatture ed offerte
\item Stima dell'IVA mensile e trimestrale
\item Gestione dei movimenti bancari
\item Gestione di spese varie non legate ad offerte
\item Diversi report sull'andamento della situazione economica
\end{itemize}

\subsubsection{\resa}
Questo modulo si occupa di gestione di progetti, tra le funzionalità offerte
figurano:

\begin{itemize}
\item Suddivisione di un progetto in attività, con scadenze fissate
\item Timer per consuntivare il tempo speso per le diverse attività
\item Possibilità di tenere traccia di incontri, scambi di email, telefonate
con i diversi clienti
\item Gestione di promemoria
\item Diversi report sull'andamento dei progetti
\end{itemize}

\subsection{\fiscoloMobile}
L'applicazione è stata pensata come funzionalità aggiuntiva per la versione
a pagamento della controparte web. \\
La vastità della controparte, la natura mobile del progetto e i limiti temporali
dello stage hanno imposto una selezione delle funzionalità da trasporre da
\fiscoloWeb{}.

\subsubsection{Obiettivi del progetto}
Nella fase iniziale di analisi sono state individuate le seguenti funzionalità:

\begin{itemize}
\item \textbf{Gestione dei promemoria}: per un/a utente mobile è sicuramente un vantaggio
poterne visualizzare una lista o segnarne di nuovi in qualsiasi situazione
\item \textbf{Gestione delle attività}: l'applicazione offre la possibilità di prendere
in carico un'attività di un progetto senza dover accedere al portale online
\item \textbf{Timer}: l'aggiunta di questa funzionalità rende più agile per le/gli utenti
la consuntivazione del tempo speso per certe attività e progetti
\item \textbf{Gestione relazioni}: anche questa funzionalità ha un senso in contesto mobile
in quanto permette ad esempio di segnarsi l'esito di una riunione appena conclusa, o di una
telefonata, senza la necessità di collegarsi al portale web
\item \textbf{Gestione costi vari}: come per il punto precedente, è vantaggioso poter tener
traccia delle spese varie (materiale di cancelleria, pasti, ecc. tutte spese che non
richiedono movimenti bancari). L'ambiente mobile permette inoltre nuove funzionalità rispetto
alla versione web come quella di fotografare uno scontrino durante la creazione di un nuovo
costo.
\item \textbf{Visualizzazione dati dell'azienda}: qualora servissero dati come partita IVA,
indirizzo, ecc.
\item \textbf{Visualizzazione report vari}: una visualizzazione in dettaglio di tutti i dati
presenti nella versione web non avrebbe molto senso per un'applicazione mobile. Si è pensato
di mostrare i dati disponibili in forma aggregata tramite grafici.
\end{itemize}

\subsubsection{Struttura dell'applicazione}
\fiscoloMobile{} è stata pensata come applicazione leggera, che non necessiti di particolari
elaborazioni interne, o di un database sul dispositivo. \\
Tutti i dati mostrati vengono recuperati da \fiscoloWeb{} tramite chiamate GET e tutti i
nuovi dati vengono salvati tramite chiamate POST. Per evitare problemi di sincronizzazione
tra mobile e web i dati vengono recuperati ad ogni apertura di pagina. \\

\subsubsection{Tecnologie utilizzate}
Per la realizzazione del progetto sono state utilizzate le seguenti tecnologie:

\begin{itemize}
\item \textbf{Phonegap\footnote{\texttt{ http://phonegap.com/}}}: framework per 
la realizzazione di applicazioni mobile
cross-platform utilizzando HTML5, Javascript e CSS (cfr. \ref{phonegap})
\item \textbf{React\footnote{\texttt{ http://facebook.github.io/react/index.html}}}: 
framework Javascript per la realizzazione di interfacce utente (cfr. \ref{react})
\item L'interfaccia con \fiscoloWeb{} è stata sviluppata tramite chiamate REST e
scambio di dati in formato JSON
\item Il design dell'interfaccia è stato pensato ispirandosi ai principi di
\textit{material design\footnote{\texttt{ https://www.google.com/design/spec/material-design/introduction.html}}} e 
realizzato utilizzando la libreria \textbf{material ui\footnote{\texttt{ http://material-ui.com/}}} per React (cfr. \ref{material-ui}).
\end{itemize}

