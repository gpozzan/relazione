\section{Esiti del progetto}\label{esiti}

Lo sviluppo dell'applicazione ha richiesto un periodo di circa quattro settimane.
Si è scelto di concentrarsi prima sulle parti più complesse e interessanti per
un'applicazione mobile, partendo dunque da tutte le funzionalità che offrono possibilità
di input da parte dell'utente e lasciando per ultime quelle di pura reportistica. \\

Le funzionalità implementate sono dunque, in ordine:

\begin{enumerate}
\item \textbf{Login}: l'applicazione offre la possibilità di autenticarsi, non quella
di registrare un nuovo account (è infatti pensata per accompagnare la versione a
pagamento di \fiscoloWeb). Per implementare questa funzionalità è stato necessario
apportare qualche modifica a \fiscoloWeb{}, nello specifico è stata aggiunta la possibilità
di associare ad ogni utente un token di autenticazione (a durata limitata) il quale viene
inviato da \fiscoloMobile{} insieme allo username negli header delle varie chiamate \gloss{REST}.
\item \textbf{Promemoria}
\item \textbf{Relazioni}: l'insieme delle relazioni era inizialmente mostrato con una
lista (cfr. \ref{schede-liste}), si è passati alle schede in vista della possibilità
futura di mostrare delle mappe associate agli indirizzi degli incontri.
\item \textbf{Costi}: la possibilità di scattare foto era inizialmente raggiungibile tramite
un normale bottone (cfr. \ref{bottoni}), si è deciso infine di utilizzare un
\textit{floating action button} per promuovere ed evidenziare questa funzionalità
prettamente mobile.
\item \textbf{Timer}: questa pagina ha subito una serie di cambiamenti in seguito a
consultazione con l'azienda. Nella prima versione veniva offerta la possibilità di creare
nuove \gloss{attività} e la presentazione generale era meno intuitiva. Si è scelto di semplificare
il più possibile in modo da offrire poche e chiare opzioni alle/agli utenti. Nello specifico
è stata individuata nel timer vero e proprio la funzionalità principale, all'apertura della
sezione viene dunque mostrato direttamente il timer delle attività prese in carico e offerta
la possibilità di prenderne in carico di nuove visualizzando la lista dei \gloss{progetti} e delle
attività relative. La funzionalità di creazione di nuove attività è stata rimossa in quanto
probabilmente non interessante per un/a utente mobile.
\item \textbf{Home page}
\end{enumerate}

Le ulteriori funzionalità previste (quelle di reportistica indicate in \ref{obiettivi})
non sono state implementate in quanto l'ultimo periodo di sviluppo è stato dedicato
alla veste grafica dell'applicazione e a tutta una serie di aspetti necessari per un
primo rilascio, ovvero:

\begin{itemize}
\item \textbf{Controllo di connessione}: l'applicazione necessita di una connessione
per funzionare, è stato dunque aggiunto un controllo relativo ed una pagina esplicativa
nel caso la rete non fosse disponibile
\item \textbf{Rilascio su Phonegap Build}: il codice è stato caricato e testato sul servizio
cloud di build di Phonegap
\item \textbf{Splash Screen}: è stata disegnata e impostata una splash screen da mostrare
al caricamento dell'applicazione
\item \textbf{Documentazione}: è stato scritto un piccolo manuale dello sviluppatore che
riassume i passi necessari ad installare i diversi strumenti necessari allo sviluppo con
Phonegap, React, Flux, ecc. e illustra brevemente l'architettura dell'applicazione e
il funzionamento e la relazione tra le sue diverse parti
\item \textbf{Transizioni}: sono stati fatti dei tentativi per inserire transizioni ed
animazioni in linea con i principi di material design (cfr. \ref{material-animation}).
Tali tentativi non hanno purtroppo avuto buon esito in quanto ci si è scontrati con
i problemi esposti nella sezione \ref{svantaggi-material-ui}.
\item \textbf{Test}: la solidità dell'applicazione è stata testata a fondo su emulatori
e dispositivi mobile, le problematiche principali individuate da questi test di sistema
sono state:
	\begin{itemize}
	\item \textit{Mancanza di pulizia delle strutture dati contenenti di una pagina}:
	sono stati necessari degli usi prolungati dell'applicazione per rendersi conto che in
	certi casi i dati ottenuti dalle chiamate GET venivano semplicemente aggiunti a delle
	liste e dunque ad ogni visita venivano mostrati raddoppiati, triplicati,
	ecc. Per risolvere questo problema è stato sufficiente ripulire le strutture dati ad ogni
	caricamento di pagina.
	\item \textit{Problemi di autenticazione}: la prima versione del login dell'applicazione
	si limitava a registrare in sessione lo username dell'utente, in realtà per il corretto
	funzionamento di \fiscoloWeb{} è necessaria anche la registrazione del nome dell'azienda
	relativa (una variabile detta \texttt{business}). Questo problema è rimasto celato per
	diverso tempo in quanto durante lo sviluppo si era quasi sempre già connessi ed autenticati
	a \fiscoloWeb{} e si è manifestato nella prima fase di testing vero e proprio utilizzando
	soltanto l'applicazione mobile. Il problema ha evidenziato quanto sia complesso lavorare
	con codice esistente (e quanto la documentazione dello stesso possa aiutare uno
	sviluppatore esterno) e l'attenzione necessaria per interazioni con ed estensioni dello
	stesso.
	\item \textit{Refresh delle pagine}: dopo l'inserzione di nuovi dati (ad esempio un nuovo
	promemoria) è giusto visualizzare immediatamente i risultati sulla lista relativa.
	L'implementazione originale richiedeva certe volte un refresh manuale della pagina per
	mostrare i risultati, per risolvere il problema è stato necessario spostare le chiamate
	relative alle funzioni di callback invocate al successo delle chiamate REST.
	\end{itemize}
\end{itemize}

Gli esiti del progetto sono stati ritenuti soddisfacenti dall'azienda in quanto sono
state sviluppate tutte le funzionalità che richiedono input da parte degli/delle utenti
e quindi offrono dei vantaggi effettivi e si è riusciti alla fine ad ottenere un prodotto
pronto per un primo rilascio.

\begin{table} [H] \centering
\begin{tabularx}{\textwidth}{|X|c|}
\hline
\textbf{Obiettivo} & \textbf{Stato} \\ \hline
Funzionalità di autenticazione & \checkmark \\ \hline
Inserimento e consultazione di promemoria & \checkmark \\ \hline
Inserimento e consultazione di relazioni & \checkmark \\ \hline
Inserimento e consultazione di costi & \checkmark \\ \hline
Funzionalità di caricamento di foto relative ai costi & \checkmark \\ \hline
Funzionalità di timer e possibilità di prendere in carico attività & \checkmark \\ \hline
Riassunto dati salienti dell'azienda & \checkmark \\ \hline
Funzionalità di reportistica & X \\ \hline
Transizioni ed animazioni in linea coi principi di material design & X \\ \hline
\end{tabularx}
\caption{Riassunto dello stato di raggiungimento degli obiettivi}
\end{table}