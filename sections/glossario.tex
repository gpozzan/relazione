\section{Glossario}

\begin{itemize}
\item \textbf{Attività}: in \fiscolo{} un'Attività è sempre parte di un \gloss{Progetto},
ha un tempo preventivato e un tempo consuntivato (inserito manualmente o calcolato tramite
il timer offerto dall'applicazione). Può essere assegnata a degli/delle utenti o presa
in carico dagli stessi/dalle stesse.

\item \textbf{Contatto}: in \fiscolo{} un Contatto rappresenta semplicemente un referente
per una o più aziende.

\item \textbf{Costo}: in \fiscolo{} un Costo è definito da una data, un importo, una
descrizione ed una categoria, tali categorie sono definite dagli utenti e possono aumentare
in modo arbitrario.

\item \textbf{Data binding}: si tratta di una connessione tra l'interfaccia utente ed
elementi della logica di business (i.e. i dati) di un'applicazione.

\item \textbf{ISO}: acronimo per \textit{Internation Organization for Standardization}.

\item \textbf{JSON}: acronimo per \textit{JavaScript Object Notation}, si tratta di un formato
di scambio dati fra applicazioni client-server.

\item \textbf{Markup}: si tratta di linguaggi standardizzati che permettono di scrivere un
testo definendone allo stesso tempo alcune proprietà strutturali e semantiche. In questa
relazione ci si riferisce al linguaggio HTML.

\item \textbf{MVC}: acronimo per \textit{Model View Controller}.

\item \textbf{MVVM}: acronimo per \textit{Model View View Model}.

\item \textbf{MVW}: acronimo per \textit{Model View Wathever}.

\item \textbf{NOOP}: si tratta di un comando la cui invocazione non ha effetti di alcun
tipo.

\item \textbf{Progetto}: in \fiscolo{} un Progetto è visto come un insieme di \gloss{attività}
che possono essere relazionate a dei \gloss{traguardi}.

\item \textbf{Promemoria}: in \fiscolo{} un Promemoria consiste di un testo (di lunghezza
qualsiasi) al quale è associata la data di creazione. Può essere attivo o concluso.

\item \textbf{Relazione}: in \fiscolo{} una Relazione è definita da una data, una tipologia
(email, incontro o telefonata), uno o più \gloss{contatti}, una descrizione ed un eventuale
indirizzo.

\item \textbf{REST}: acronimo per \textit{REpresentational State Transfer}, si riferisce a
interfacce che permettono di trasmettere dati su HTTP senza necessità di sessioni o di
mantenimento di stato.

\item \textbf{Tap}: modalità di interazione mobile che consiste nella rapida pressione
di un'area di schermo con il dito.

\item \textbf{Touch ripple}: nome dato a una reazione visiva associata al tocco di un
particolare componente, l'effetto è quello di un'onda che si propaga come accade sulle
superfici liquide.

\item \textbf{Traguardo}: si tratta in sostanza di quelle che nel gergo dell'Ingegneria del
Software vengono chiamate \textit{milestone}. Sono scadenze di una certa importanza come
rilasci, revisioni di avanzamento, ecc.

\item \textbf{Usabilità}: viene definita dall'\gloss{ISO} come \textit{l'efficacia, l'efficienza
e la soddisfazione con le quali determinati utenti raggiungono determinati obiettivi in
determinati contesti}. Uno studio dell'usabilità di un'applicazione o di un sito web ne valuta
dunque l'intuitività e la possibilità di un utilizzo sereno da parte degli/delle utenti.
\end{itemize}